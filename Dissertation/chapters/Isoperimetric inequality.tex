\documentclass[../main.tex]{subfiles}

\begin{document}

In this chapter we shall find an inequality for Gaussian spaces analogous to the isoperimetric inequality on Euclidean spaces. It turns out that the subspaces that maximise area for a fixed surface measure in a Gaussian space is a half space. Our proof will employ a geometric method, using Gaussian $k$-symmetrizations, which we shall introduce next.

\section{Gaussian symmetrization}
Throughout this section, we consider the probability space $\del{\R^n,\Bor(\R^n),\gamma_n}$, where $\gamma_n$ is the standard Gaussian distribution on $\R^n$.
\begin{definition}
\label{def:Gauss_symm}
Let $1\leq k\leq n$, $L$ be a linear subspace of $\R^n$ of dimension $n-k$, and $e\in L^\perp$ be a unit vector. We define the Gaussian $k$-symmetrization to be the mapping $S(L,e)$ on closed subsets $A\subset\R^n$ as follows \maths{S(L,e)(A) := \bigcup_{x\in L}\del{\cbr{y:\abrac{y,e}\geq r}\cap\del{x + L^\perp}},} where $r$ is such that \maths{\gamma_k\del{\cbr{y:\abrac{y,e}\geq r}\cap\del{x + L^\perp}} = \gamma_k\del{A\cap\del{x + L^\perp}},} and $\gamma_k$ is the standard $k$-dimensional Gaussian measure the $k$-dimensional affine subspace $x + L^\perp$.
\end{definition}
This is quite an abstract definition, but it is quite easy to picture in the case that $n=1$. For example, take $n=1$, $L = \cbr{0}$, and $A$ a closed interval in $\R$. Then setting $\alpha = \gamma_1(A)$, we have $S(L,e)(A) = \intoo{\Phi^{-1}(1-\alpha),\infty}$, where $\Phi(x) = \gamma_1\intoo{-\infty,x}$, see \figref{fig:gauss_symm}. 
\vspace{0.5cm}
\begin{figure}[H]
    \centering
    \begin{tikzpicture}
    \begin{axis}[
      no markers, domain=-3:3, samples=100,
      axis lines*=left, xlabel=$x$,
      every axis y label/.style={at=(current axis.above origin),anchor=south},
      every axis x label/.style={at=(current axis.right of origin),anchor=west},
      height=5cm, width=12cm,
      xtick={-1.5, -0.5, 0, 0.7007}, ytick=\empty,
      enlargelimits=false, clip=false, axis on top,
      grid = major
      ]
      \addplot [fill=blue!20, draw=none, domain=-1.5:-0.5] {gauss(0,1)} \closedcycle;
      \addplot [very thick,cyan!50!black] {gauss(0,1)};
      \addplot [fill=red!20, draw=none, domain=0.7007:3] {gauss(0,1)} \closedcycle;
    
    
        \draw [yshift=-0.6cm, latex-latex](axis cs:-1.5,0) -- node [fill=white] {$A$} (axis cs:-0.5,0);
        \draw [yshift=-0.6cm, latex-latex](axis cs:0.7007,0) -- node [fill=white] {$S(L,e)(A)$} (axis cs:3,0);
    \end{axis}
    
    \end{tikzpicture}
    \caption{Gaussian symmetrization in the case that $n=1$, $L = \cbr{0}$, $A = \intcc{-1.5,-0.5}$. $\gamma_1(A)$ is given by the blue area under the density curve of $\gamma_1$. Then $S(L,e)(A) = \intoo{r,\infty}$, where $r = \Phi^{-1}\del{1-\gamma_1(A)} = 0.7007$. Importantly the red area is equal to the blue area.}
    \label{fig:gauss_symm}
\end{figure}
This idea can be extended to $\R^n$ in the sense that a Gaussian $n$-symmetrization, $S(L,e)(A)$, will give a halfspace with $\gamma_n$ measure equal to $\gamma_n(A)$.

We then have the following properties of Gaussian symmetrizations.
\begin{proposition}
\label{prop:Gauss_symm_properties}
For arbitrary closed sets $A,B\in\R^n$, one has:
\begin{enumerate}
    \item If $A\subset B$, then $S(L,e)(A)\subset S(L,e)(B)$;
    \item $S(L,e)(A^C) = S(L,-e)(A)^C$;
    \item For all $v\in L$ we have $S(L,e)(A+v) = S(L,e)(A)+v$;
    % \item $S(L,e)(A) = S(L,e)(A) + \del{L + \R^1e}^\perp$, and $S(L,e)(A) +\lambda e\subset S(L,e)(A)$ for all $\lambda\geq0$;
    \item If $\cbr{A_i}$ is an increasing sequence of open sets such that $A = \bigcup_i A_i$, then \maths{S(L,e)(A) = \bigcup_i S(L,e)(A_i).}
    % \item If $\gamma_n(B) = \gamma_n(B+L^\perp)$, then $\gamma_n(A\cap B) = \gamma_n\del{S(L,e)(A)\cap B}$. 
\end{enumerate}
\end{proposition}
\begin{proof}
\begin{enumerate}
    \item Suppose that $A\subset B$, and fix $x\in L$, then notice that $A\cap\del{x + L^\perp}\subset B\cap\del{x + L^\perp}$. In particular, the $r_B$ appearing in the definition $S(L,e)(B)$ is larger than or equal to the corresponding $r_A$ appearing in the definition of $S(L,e)(A)$. This implies that $S(L,e)(A)\cap\del{x + L^\perp}\subset S(L,e)(B)\cap\del{x + L^\perp}$, and then taking the union over all $x\in L$ gives the result.
    \item Notice for a fixed $x\in L$ that since $\gamma_k\del{A\cap\del{x + L^\perp}} + \gamma_k\del{A^C\cap\del{x + L^\perp}} = 1$, therefore, we have that $S(L,-e)(A)\cap\del{x + L^\perp}$ and $S(L,e)(A^C)\cap\del{x + L^\perp}$ are halfspaces of $x + L^\perp$ in opposite directions, such that the sum of their $\gamma_k$ measures is $1$. This implies that $\del{S(L,-e)(A)\cap\del{x + L^\perp}}^C = S(L,e)(A^C)\cap\del{x + L^\perp}$. Again, taking the union over all $x\in L$ gives the result.
    \item Since for fixed $v\in L$ and fixed $x\in L$, we have that \maths{\gamma_k\del{A\cap\del{x + L^\perp}} = \gamma_k\del{(A+v)\cap\del{x + v + L^\perp}},} therefore, we have \maths{S(L,e)(A)\cap\del{x + L^\perp} + v = S(L,e)(A+v)\cap\del{x + v+ L^\perp},} and taking the union over all $x\in L$ gives the result. 
    \item For fixed $x\in L$, since we have that $\cbr{A_i}$ is an increasing sequence we know that \maths{\gamma_k\del{\bigcup_iA_i\cap\del{x + L^\perp}} = \lim_{i\to\infty}\gamma_k\del{A_i\cap\del{x + L^\perp}}.} This implies that \maths{S(L,e)\del{\bigcup_iA_i}\cap\del{x + L^\perp} = \bigcup_iS(L,e)(A_i)\cap\del{x + L^\perp},} and taking the union over all $x\in L$ gives the result.
\end{enumerate}
\end{proof}

\section{The Gaussian isoperimetric inequality}
There are a number of proofs of the Gaussian isoperimetric inequality, in particular, in the 1970's both Borell \cite{Borell1975} and Sudakov \cite{Sudakov1974} were able to prove the isoperimetric inequality. In the 1980's Ehrhard was able to prove the inequality using Gaussian symmetrizations \cite{Ehrhard1983}. In the same paper, Ehrhard derived an inequality similar to the Brunn-Minkowski inequality in Euclidean space, which offers another proof of the Gaussian isoperimetric inequality \cite{Latala2002}. It is the latter argument that we shall discuss in this dissertation.

We first introduce Ehrhard's inequality.
\begin{theorem}[Ehrhard's inequality]
\label{the:ehrhards_inequality}
Let $A$ and $B$ be two Borel sets in $\R^n$. Then for all $\lambda\in\intcc{0,1}$
\begin{equation}
\label{eq:ehrhard}
    \Phi^{-1}\cbr{\gamma_n\del{\lambda A + (1-\lambda)B}}\geq\lambda\Phi^{-1}\cbr{\gamma_n\del{A}} + (1-\lambda)\Phi^{-1}\cbr{\gamma_n\del{B}}.
\end{equation}
\end{theorem}
In the statement of the original theorem by Antoine Ehrhard \cite{Ehrhard1983}, we require that $A,B\subset\R^n$ are convex sets, however, above we state the inequality for any measurable sets $A,B$ in $\R^n$. The proof of this, however, is outside of the scope of this dissertation, but can be found in Borell's 2003 paper `The Ehrhard inequality' \cite{Borell2003}. Below, we present the proof under the assumption that $A$ and $B$ are convex sets.

\begin{proof}
First, let us consider $A+e_{n+1}$ and $B$ to be sets in $\R^{n+1}$, where $\cbr{e_i}_{1\leq i\leq n+1}$ is the standard basis for $\R^{n+1}$. Then take $C$ to be the convex hull of $A + e_{n+1}$ and $B$, then we may define \maths{C_\lambda := \R^n\cap \del{C-\lambda e_{n+1}} = \lambda A + (1-\lambda)B.} We now define the function \maths{f(\lambda) = \Phi^{-1}\cbr{\gamma_n\del{C_\lambda}}.} In particular notice that \eqref{eq:ehrhard} is equivalent to 
\begin{equation}
\label{eq:new_ehrhard}
    f(\lambda)\geq\lambda f(1) + (1-\lambda)f(0).
\end{equation}

To show that \eqref{eq:new_ehrhard} holds, let us consider the Gaussian $n$-symmetrization $S(L,e)$ with $L = \R^1e_{n+1}$ and $e\in\cbr{e_i}_{1\leq i\leq n}$. Then we have that $r_\lambda = -f(\lambda)$ to be the $r$ appearing in the definition of Gaussian symmetrizations. Notice that $S(L,e)(C)$ is convex, therefore \maths{\lambda\del{\del{e_{n+1} +\R^n}\cap S(L,e)(C)} + (1-\lambda)\del{\R^n\cap S(L,e)(C)}\subset\del{\lambda e_{n+1} + \R^{n}}\cap S(L,e)(C).} Specifically, this implies that $\lambda\intoo{r_1,\infty} + (1-\lambda)\intoo{r_0,\infty} \subset \intoo{r_\lambda,\infty}$, or equivalently $r_\lambda\leq \lambda r_1 + (1-\lambda)r_0$. This in turn implies that \eqref{eq:new_ehrhard} holds, thus completing the proof.
\end{proof}

Finally, we arrive at the Gaussian isoperimetric inequality.
\begin{theorem}
\label{the:gaussian_isoperimetric_inequality}
Let $\gamma_n$ be the standard Gaussian measure on $\R^n$, and let $U$ be the closed unit ball in $\R^n$. Then for any measurable set $A\in\R^n$, and any $\epsilon\geq0$, the following holds 
\begin{equation}
    \label{eq:iso_inequality}
    \Phi^{-1}\cbr{\gamma_n\del{A+\epsilon U}}\geq \Phi^{-1}\cbr{\gamma_n\del{A}} + \epsilon.
\end{equation}
\end{theorem}
\begin{proof}
This statement follows almost immediately from \thmref{the:ehrhards_inequality}. Suppose that $A$ is a measurable set in $\R^n$, then we consider the following \maths{\Phi^{-1}\cbr{\gamma_n\del{A + \epsilon U}} &= \Phi^{-1}\cbr{\gamma_n\del{\lambda(\lambda^{-1}A) + (1-\lambda)(1-\lambda)^{-1}\epsilon U}} \\ &\geq \lambda\Phi^{-1}\cbr{\gamma_n\del{\lambda^{-1}A}} + (1-\lambda)\Phi^{-1}\cbr{\gamma_n\del{(1-\lambda)^{-1}\epsilon U}}.} Taking $\lambda\to 1$, the right hand side goes to $\Phi^{-1}\cbr{\gamma_n(A)} + \epsilon$, giving \eqref{eq:iso_inequality}.
\end{proof}

It is not exactly obvious how \thmref{the:gaussian_isoperimetric_inequality} is related to an isoperimetric inequality, or indeed, what set maximises the volume to surface area ratio under the standard Gaussian measure. In order to understand this inequality, let us rewrite it as 
\begin{equation}
    \label{eq:gauss_iso_rewrite}
    \gamma_n\del{A + \epsilon U}\geq\Phi\del{a + \epsilon},
\end{equation}
where $a = \Phi^{-1}\cbr{\gamma_n(A)}$. However, notice that $\Phi\del{a + \epsilon}$ is then equal to $\gamma_n\del{H + \epsilon U}$, where $H$ is a halfspace such that $\gamma_n(H) = \gamma_n(A)$. Such an $H$ exists since $\gamma_n$ is a bounded measure for any $n$. Therefore, we can rewrite \eqref{eq:gauss_iso_rewrite} as 
\begin{equation}
    \label{eq:gauss_iso_rerewrite}
    \gamma_n\del{A + \epsilon U}\geq\gamma_n\del{H + \epsilon U}.
\end{equation}

We may then define the \emph{surface measure}, $\gamma_n^\text{surf}$, of a set $A\in\R^n$ as the following limit \maths{\gamma_n^\text{surf}(A) = \lim_{\epsilon\to0}\frac{\gamma_n(A+\epsilon U)-\gamma_n(A)}{\epsilon}.} Intuitively, this corresponds to how much the measure of the set $A$ changes by adding $\epsilon U$, which corresponds to the measure of the surface of the set. It is then evident from \eqref{eq:gauss_iso_rerewrite} that for arbitrary $A\in\R^n$ and a halfspace $H\in\R^n$ such that $\gamma_n(A) = \gamma_n(H)$, we have \maths{\gamma_n^\text{surf}(A) = \lim_{\epsilon\to0}\frac{\gamma_n\del{A+\epsilon U} - \gamma_n(A)}{\epsilon}\geq\lim_{\epsilon\to0}\frac{\gamma_n\del{H + \epsilon U}-\gamma_n(H)}{\epsilon} = \gamma_n^\text{surf}(H).} In other words, halfspaces minimise surface measure. One can compare this to the isoperimetric inequality in Euclidean space which states that spheres minimise surface measure.







\end{document}